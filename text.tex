\section*{\CilPrace}

Hlavním cílem této práce byl návrh funkcionalit přidaných do integrovaného studijního informačního systému InSIS a následná implementace těchto navržených přidaných funkcionalit v~podobě rozšíření do webových prohlížečů společně s~podpůrným serverem pro ukládání dat uživatelů rozšíření. 

Druhým dílčím cílem práce bylo představení a srovnání dostupných technologií pro obě zmíněné části softwarového řešení a následný výběr použité technologie. Bylo představeno široké spektrum technologií společně s~celou řadou programovacích jazyků. Dále byly vysvětleny omezení běhových prostředí a rozdíl mezi požadavky, které byly kladeny na výběr obou částí.  

Posledním dílčím cílem práce bylo vyhodnocení zpětné vazby od uživatelů webového rozšíření získané metodou dotazníkového šetření. Byly stanoveny výzkumné otázky, na které toto dotazníkové šetření odpovídá pomocí vyhodnocení kvantitativních otázek. Respondenti měli dále možnost v~rámci dotazníkového šetření odpovídat na kvantitativní otázky, které mohou sloužit pro vymezení podnětů pro další zlepšení tohoto projektu do budoucna. 

\section*{\PouziteMetody}

V~práci byly použity metody dotazníkového šetření, komparace, analýzy a syntézy. Dotazníkové šetření bylo využito v~poslední části práce pro sběr zpětné vazby od ostatních uživatelů naprogramovaného rozšíření do webových prohlížečů. Metody komparace, analýzy a syntézy byly využity při výběru technologií pro implementaci, návrhu přidaných funkcionalit a popisu implementačních detailů obou částí řešení.    

Dotazník pro sběr zpětné vazby byl vytvořen pomocí platformy Google Forms. Rozšíření do webových prohlížečů bylo implementováno v~programovacím jazyce TypeScript společně s~knihovnou React.js a sadou komponent Chakra UI. Podpůrný webový server pro rozšíření byl implementován v~jazyce Kotlin v~rámci aplikačního rámce Spring Boot. Pro ukládání dat uživatelů byl použit systém řízení báze dat PostgreSQL s~připojením prostřednictvím softwarové knihovny R2DBC. Pro ukládání a správu verzí zdrojového kódu byla zvolena platforma GitLab. V~rámci této platformy dochází k~automatizovanému sestavování balíčků rozšíření a spouštění strojových testů v~rámci definované GitLab CI/CD pipeline.

\section*{\Vysledky}

Byly navrženy a následně implementovány tři funkcionality přidané do informačního systému InSIS a jedna podpůrná funkcionalita, kterou je přihlášení do webového rozšíření pomocí školní emailové adresy.  Implementované rozšiřující funkcionality jsou vylepšené zobrazení rozvrhu s~možností přidání poznámek k~jednotlivým hodinám a přepínání mezi týdny s~automatickým zpracováním volných dní a dnů konání blokových akcí. Druhou přidanou funkcionalitou je přidání možnosti si nechat zasílat upozornění emailem na odevzdávárny s~blížícím se datem uzavření a jejich export do externího kalendáře. Poslední přidanou funkcionalitou je zobrazení náhledu rozvrhu při registraci rozvrhových akcí společně s~automatickou detekcí kolizí, která zlepšuje produktivitu uživatelů při práci s~tímto modulem informačního systému InSIS.

Výsledkem práce je funkční, koherentní softwarové řešení v~podobě rozšíření do webových prohlížečů s~podporou pro prohlížeče založených na technologii Chromium a Mozilla Firefox. Toto rozšíření je publikováno na internetových obchodech Google Web Store a Firefox Addons, na kterých má v~době psaní přes 300 instalací a 120 aktivních uživatelů. Součástí řešení je i již zmíněný podpůrný webový server, který je nasazen s~pomocí technologie Docker u~poskytovatele Digital Ocean. Server je navržen tak, aby v~případě potřeby většího výpočetního výkonu bylo možné snadno tuto aplikaci škálovat s~minimálním zásahem do nasazené infrastruktury.

\section*{\PrinosAutora}

Vlastním přínosem autora bylo navržení funkcionalit, které předtím nebyly adresovány a které vychází z~předchozí analýzy v~podobě konzultací s~ostatními studenty na Vysoké škole ekonomické v~Praze. Tyto navržené funkcionality byly následovně implementovány a bylo provedeno dotazníkové šetření pro ověření shody podoby implementace se skutečnými potřebami uživatelů a počáteční analýzou. 

Motivací pro tuto práci byla předchozí zkušenost autora a ostatních studentů s~používáním informačního systému InSIS. Na základě této předchozí zkušenosti byly vymezeny slabá místa s~potenciálem pro zlepšení uživatelské produktivity při používání. Z~odpovědí na kvalitativní otázky v~rámci dotazníkového šetření vyplynulo, že uživatelé rozšíření jsou s~jeho současnou podobou spokojení a že je pro ně používání tohoto rozšíření přínosné.

% víc vlastních poznatků n shit a ne tolik ostatní